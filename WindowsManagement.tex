\documentclass[12pt]{article}
\usepackage[a4paper, margin=2.5cm]{geometry}
\usepackage{fontspec}
\usepackage{graphicx}
\usepackage{setspace}
\usepackage{titlesec}

\setmainfont{Times New Roman}

% === STYLE SETTINGS ===
\setstretch{1.5}
\titleformat{\section}{\bfseries\large}{\thesection.}{0.5em}{}

% === TITLE INFO ===
\title{\textbf{Analisis Manajemen Sistem pada Sistem Operasi Windows}}
\author{Muhamad Rizal -- 11241049 \\ 
Mata Kuliah: IF2514303 -- Sistem Operasi}
\date{}

\begin{document}

\maketitle

\section{Manajemen Proses}
Pada sistem operasi Windows, manajemen proses bertugas untuk mengatur pembuatan, penjadwalan, dan penghentian proses. 
Windows menggunakan \textit{Windows Scheduler} untuk menentukan proses mana yang dijalankan oleh CPU. 
Windows menerapkan konsep \textit{multitasking preemptive}, yang berarti CPU dapat mengganti proses sebelum proses yang sedang dijalankan selesai agar sistem terasa lebih responsif.

Adapun beberapa komponen penting pada manajemen proses adalah sebagai berikut:
\begin{itemize}
    \item \textbf{Process Control Block (PCB)}  Berfungsi untuk menyimpan informasi penting seperti \textit{Process ID, Process State, CPU Contex, Memory Management, Scheduling Information, I/O Infromation, Security Info}.
    \item \textbf{Windows Task Manager}  Alat yang digunakan untuk memantau dan mengelola proses yang sedang berjalan. Melalui \textit{Task Manager user} dapat memantau kinerja sistem secara \textit{Realtime}, termasuk
    \textit{CPU Usage, RAM usage, Storage Activity, Network Activity,} serta status dari masing-masing proses. \textit{User} juga bisa mengakhiri dan mengatur prioritas proses.
    \item \textbf{Windows Service}  berfungsi sebagai kumpulan proses \textit{background} 
    yang berjalan tanpa antarmuka pengguna (\textit{non-interactive}). 
    Contohnya termasuk \textit{Windows Update Service, Print Spooler,} dan \textit{Windows Defender Service}. 
    Layanan-layanan ini dikelola oleh \textit{Service Control Manager} (SCM), yang bertanggung jawab untuk memulai, menghentikan, atau mengonfigurasi \textit{service} sesuai kebutuhan sistem. 
    Mereka sering berjalan dengan hak akses tinggi (\textit{system-level privileges}) agar bisa mengakses komponen kernel atau \textit{hardware}.
\end{itemize}

\section{Manajemen Memori}
Manajemen memori dalam sistem operasi Windows memiliki tugas untuk 
mengalokasikan dan mengatur ruang memori untuk proses agar tidak saling bertabrakan.
Dalam penerapannya Windows menggunakan \textit{Virtual Memory System} yang memetakan \textit{Physical Memory} (RAM) dan \textit{Virtual Memory} di \textit{page file}

adapun Mekanisme yang digunakan oleh Windows untuk melakukan manajemen memori yaitu:
\begin{itemize}
    \item \textbf{Paging} yaitu proses memecah memori menjadi blok-blok kecil yang 
    disebut \textit{pages}. Sistem operasi kemudian memetakan halaman-halaman ini ke alamat 
    virtual, sehingga aplikasi dapat menggunakan lebih banyak memori daripada yang 
    tersedia secara fisik di RAM. Dengan \textit{virtual memory}, Windows mampu menjaga agar 
    sistem tetap responsif bahkan saat banyak aplikasi berjalan bersamaan.
    \item \textbf{Page File}
    \item \textbf{Memory Protection}
    \item \textbf{Dynamic Link Library (DLL) Sharing}
\end{itemize}
\section{Manajemen Storage}

\section{Manajemen I/O}

\section{Manajemen Keamanan}

\end{document}
