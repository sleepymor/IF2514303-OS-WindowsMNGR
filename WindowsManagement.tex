\documentclass[12pt]{article}
\usepackage[a4paper, margin=2.5cm]{geometry}
\usepackage{fontspec}
\usepackage{graphicx}
\usepackage{setspace}
\usepackage{titlesec}

\setmainfont{Times New Roman}

% === STYLE ===
\setstretch{1.5}
\titleformat{\section}{\bfseries\large}{\thesection.}{0.5em}{}

% === TITLE INFO ===
\title{\textbf{Analisis Manajemen Sistem Pada Sistem Operasi Windows}}

\author{Muhamad Rizal - 11241049 \\
Mata Kuliah: IF2514303 - Sistem Operasi}
\begin{document}

\maketitle

\section{Manajemen Proses}
Pada sistem operasi Windows manajemen proses bertugas untuk mengatur pembuatan, penjadwalan, dan penghentian proses\@. Windows menggunakan Windows Scheduler untuk menentukan proses mana yang di jalankan oleh CPU\@. 
Windows menggunakan multitasking preemptive, yang berarti CPU dapat mengganti proses sebelum prosses yang sedang di jalankan selesai, agar sistem terasa lebih responsif.

Adapun beberapa komponen penting yaitu:
    \begin{description}
        \item[Process Control Block]
        \item[Windows Task Manager]
        \item[Windows Service]   
    \end{description}
\section{Manajemen Memori}
\section{Manajemen Storage}
\section{Manajemen I/O}
\section{Manajemen Keamanan}


\end{document}